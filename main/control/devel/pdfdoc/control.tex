\input texinfo   @c -*-texinfo-*-
@c %**start of header
@setfilename control.info
@settitle Octave Control Systems Package
@afourpaper
@set VERSION 2.2.1
@finalout
@c @afourwide
@c %**end of header

@c %*** Start of TITLEPAGE
@titlepage
@title control @value{VERSION}
@subtitle Control Systems Package for GNU Octave
@author Lukas F. Reichlin
@page
@vskip 0pt plus 1filll
Copyright @copyright{} 2009-2011, Lukas F. Reichlin @email{lukas.reichlin@@gmail.com}

This manual is generated automatically from the texinfo help strings
of the package's functions.

Permission is granted to make and distribute verbatim copies of
this manual provided the copyright notice and this permission notice
are preserved on all copies.

Permission is granted to copy and distribute modified versions of this
manual under the conditions for verbatim copying, provided that the entire
resulting derived work is distributed under the terms of a permission
notice identical to this one.

Permission is granted to copy and distribute translations of this manual
into another language, under the same conditions as for modified versions.
@page
@chapheading Preface
The @acronym{GNU} Octave control package from version 2 onwards was
developed by Lukas F. Reichlin and is based on the proven open-source
library @acronym{SLICOT}. This new package is intended as a replacement
for control-1.0.11 by A. Scottedward Hodel and his students.

@sp 5
@subheading Acknowledgments
The author is indebted to several people and institutions who helped
him to achieve his goals. I am particularly grateful to Luca Favatella
who introduced me to Octave development as well as discussed and revised
my early draft code with great patience. My continued support from the
@acronym{FHNW} University of Applied Sciences of Northwestern Switzerland,
where I could work on the control package as a semester project, has also
been important. Furthermore, I thank the @acronym{SLICOT} authors
Peter Benner, Vasile Sima and Andras Varga for their advice.


@sp 5
@subheading Using the help function
Some functions of the control package are listed with a leading @code{@@lti/}.
This is only needed to view the help text of the function, e.g. @w{@code{help norm}}
shows the built-in function while @w{@code{help @@lti/norm}} shows the overloaded
function for @acronym{LTI} systems. Note that there are @acronym{LTI} functions
like @code{pole} that have no built-in equivalent.

When just using the function, the leading @code{@@lti/} must @strong{not} be typed.
Octave selects the right function automatically. So one can type @w{@code{norm (sys, inf)}}
and @w{@code{norm (matrix, inf)}} regardless of the class of the argument.
@end titlepage
@c %*** End of TITLEPAGE

@contents
@c @chapter Function Reference
@include functions.texi

@end
@bye
