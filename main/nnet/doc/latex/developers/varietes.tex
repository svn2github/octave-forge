\section{Varietes}
\label{chap:intro:sec:varietes}
\subsection{Object-oriented programming}
Some of the functions for Octave will have not the same name, like the matlab ones has.
This difference is because the object-oriented programming technology. The object-oriented functions will have a name postfix in Octave.\\
As example: \textit{isposint\_netw\_priv} means, \textit{isposint} is a private function from the matlab function \textit{network}.\\
This is a kind of "`simulation"' for the object-oriented programming technology.\\

\subsubsection{Data-Typ \textit{network} and m-file \textit{subsasgn}}
Matlab has a further data type \textit{network}. A basic neural network will be initialized in a
structure and than changed to the \textit{network} type with the \textit{class} command. The class command from Octave doesn't support the creation of this network type!
In the Matlab m-file \textit{network.m} on row 256, the network type will be created! From this moment, structure subscription assignment will call the file subsasgn from the \textit{@network} directory. Back in the \textit{newff} m-file on row 134, the @network subsasgn will be called the first time.

\paragraph{newff row 134; net.biasConnect} net=subsasgn(net,subscripts,v) will hold the following values:
\begin{itemize}
	\item net: the whole net structure
	\item subscripts: 1x1 structure;\\
		subscripts.type = '.'\\
		subscripts.subs = 'biasConnect'
	\item v: 2x1 double [1; 1]
\end{itemize}

\paragraph{newff row 135; net.inputConnect} net=subsasgn(net,subscripts,v) will hold the following values:
\begin{itemize}
	\item net: the whole net structure
	\item subscripts: 1x2 structure;\\
		subscripts(1).type = '.'\\
		subscripts(1).subs = 'inputConnect'\\
		subscripts(2).type = '()'\\
		subscripts(2).subs = [1] [1]
	\item v: 1x1 double 1
\end{itemize}
\textcolor{red}{bis hier sollte es erledigt sein...!}

\paragraph{newff row 137; net.layerConnect} net=subsasgn(net,subscripts,v) will hold the following values:
\begin{itemize}
	\item net: the whole net structure
	\item subscripts: 1x1 structure;\\
		subscripts(1).type = '.'\\
		subscripts(1).subs = 'layerConnect'
	\item v: 2x2 logical [0 0; 1 0]
\end{itemize}

\paragraph{newff row 138; net.outputConnect} net=subsasgn(net,subscripts,v) will hold the following values:
\begin{itemize}
	\item net: the whole net structure
	\item subscripts: 1x2 structure;\\
		subscripts(1).type = '.'\\
		subscripts(1).subs = 'outputConnect'\\
		subscripts(2).type = '()'\\
		subscripts(2).subs = '[2]'
	\item v: 1x1 double = 1
\end{itemize}

\paragraph{newff row 139; net.targetConnect} net=subsasgn(net,subscripts,v) will hold the following values:
\begin{itemize}
	\item net: the whole net structure
	\item subscripts: 1x2 structure;\\
		subscripts(1).type = '.'\\
		subscripts(1).subs = 'targetConnect'\\
		subscripts(2).type = '()'\\
		subscripts(2).subs = '[2]'
	\item v: 1x1 double = 1
\end{itemize}


\subsection{Data types}

\begin{table}
	\centering
	\begin{tabular}{c c c}
		\toprule
									& Matlab										& Octave\\
		\midrule
  								& double 											& double \\
									& int8  											& \\
									& int16 											& \\
									& int32 											& \\
									& int64 											& \\
			Numeric			& single											& \\
									& uint8 											& \\
									& uint16											& \\
									& uint32 											& \\ 
									& uint64 											& \\ 
									& complex 										& complex\\
		\midrule
			characters	& x														&  \\
		\midrule
			string			&															& \\
		\midrule
			cell				&	x														& x \\
		\midrule
			structure		& x 												  & x  \\
		\bottomrule
	\end{tabular}
\end{table}