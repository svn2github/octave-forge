% Pr�ambel
\documentclass[a4paper,openany]{report}


\usepackage{a4wide}
\usepackage[ansinew]{inputenc}
\usepackage[T1]{fontenc}
\RequirePackage{ifpdf}
%\usepackage[ngerman]{babel}


\usepackage{hyperref}
	\hypersetup{%
  colorlinks=true,   % aktiviert farbige Referenzen
  pdfpagemode=None,  % PDF-Viewer startet ohne Inhaltsverzeichnis et.al.
  pdfstartview=FitH, % PDF-Viewer benutzt beim Start bestimmte Seitenbreite
  %linkbordercolor=111,
  % citebordercolor=111,
  citecolor=blue,
  linkcolor=blue}

\ifpdf
  \usepackage[pdftex]{graphicx}
	  \DeclareGraphicsExtensions{.pdf}
\else
  \usepackage[dvips]{graphicx}
	  \DeclareGraphicsExtensions{.eps}
\fi
%
%\usepackage[pdftex]{hyperref}
%	\hypersetup{%
%  colorlinks=true,   % aktiviert farbige Referenzen
%  pdfpagemode=None,  % PDF-Viewer startet ohne Inhaltsverzeichnis et.al.
%  pdfstartview=FitH, % PDF-Viewer benutzt beim Start bestimmte Seitenbreite
%  %linkbordercolor=111,
%  % citebordercolor=111,
%  citecolor=blue,
%  linkcolor=blue}
	
\usepackage{fancyhdr}
%\usepackage{supertabular}
\usepackage{booktabs}
%\usepackage{longtable}
\usepackage[dvips]{rotating}
\usepackage{multirow}
\usepackage{multicol}

\usepackage{color}
\usepackage{amsmath}
\usepackage{alltt}
%\usepackage{array}
%\usepackage{colortbl}

%%%%%%%%%%%%%%%% wird ben�tigt, um Farben f�r Latex-Seiten von Highlight zu definieren...
\newcommand{\hlstd}[1]{\textcolor[rgb]{0,0,0}{#1}}
\newcommand{\hlnum}[1]{\textcolor[rgb]{0.75,0,0.35}{#1}}
\newcommand{\hlesc}[1]{\textcolor[rgb]{0.42,0.35,0.8}{#1}}
\newcommand{\hlstr}[1]{\textcolor[rgb]{0.75,0,0.35}{#1}}
\newcommand{\hldstr}[1]{\textcolor[rgb]{0.75,0,0.35}{#1}}
\newcommand{\hlslc}[1]{\textcolor[rgb]{0.25,0.38,0.56}{#1}}
\newcommand{\hlcom}[1]{\textcolor[rgb]{0.25,0.38,0.56}{#1}}
\newcommand{\hldir}[1]{\textcolor[rgb]{0.8,0,0.8}{#1}}
\newcommand{\hlsym}[1]{\textcolor[rgb]{0,0,0}{#1}}
\newcommand{\hlline}[1]{\textcolor[rgb]{0.25,0.38,0.56}{#1}}
\newcommand{\hlkwa}[1]{\textcolor[rgb]{0.65,0.16,0.16}{\bf{#1}}}
\newcommand{\hlkwb}[1]{\textcolor[rgb]{0.18,0.55,0.34}{\bf{#1}}}
\newcommand{\hlkwc}[1]{\textcolor[rgb]{0.15,0.37,0.93}{\bf{#1}}}
\newcommand{\hlkwd}[1]{\textcolor[rgb]{0.32,0.11,0.78}{#1}}
\definecolor{bgcolor}{rgb}{1,0.85,0.73}
\oddsidemargin -3mm
\textwidth 165,2truemm
\topmargin 0truept
\headheight 0truept
\headsep 0truept
\textheight 230truemm
%%%%%%%%%%%%%%%%%%%%%%%%%%%%%%%%%%%%%%%%%%%%%%%%%%%%%%%%%%%%%%%%%%%%%%%%%%%%%%%%%

\clubpenalty = 10000
\widowpenalty = 10000 \displaywidowpenalty = 10000

\definecolor{hellgrau}{gray}{0.95}
\definecolor{dunkelgrau}{gray}{0.55}

\definecolor{brown}{rgb}{0.75,0.004,0.3}


\renewcommand{\headrulewidth}{0pt} % keine Trennlinie
\renewcommand{\footrulewidth}{0pt} % keine Trennlinie
%\addtolength{\headheight}{39pt}

%\setlength{\footskip}{20pt}

%\nointend
%%%%%%%%%%%%%%%%%%%%%%%%%%%%%%
% start text here!!


\begin{document}
\pagenumbering{roman}
% start text here!!

\title{A neural network package for Octave\\
		User's Guide \\
				Version: 0.1.9}

\author{Michel D. Schmid}
\maketitle


\tableofcontents
%\chapter*{Notation}

% diese Aufz�hlung bitte 2-Spaltig
\begin{multicols}{2}[]

\section*{Kapitel }

\begin{tabbing}
$T_t$ ~~~~~~~~~~~~~~~~\=erster Tabulator~~~~~~~~~~~~~~~~~~\=\kill
$T_t$    \> gesamte Totzeit						  \\
$T_s$	   \> Data Bus Sampling Time			\\
$T_c$	   \> Controller Sampling Time		\\
$T_d$	   \> Residual Delay Time					\\
$k$      \> aktuelle Sampling-Nummer 		\\
$\eta$   \> Spindel-Wirkungsgrad 				\\
$Acmd$   \> Soll-Beschleunigung 			  \\
$Vcmd$   \> Soll-Geschwindigkeit 			 	\\
$Scmd$   \> Soll-Position  							\\
$iValue$ \> Stromamplitude 							\\
$Iact$	 \>	Stromamplitdue 							\\
\end{tabbing}


\end{multicols}
\pagenumbering{arabic}

\chapter{Introduction}
This documentation isn't a well defined and structured docu for the neural network toolbox.
It's more a \textit{collection of my ideas and minds}.

\section{Installed system}
I'm developing and testing the actual version of the neural network toolbox on following 
program versions:

\begin{itemize}
  \item Octave 2.9.5
  \item octave-forge-2006.01.28
  \item OctPlot svn version
\end{itemize}


\section{Version numbers of the neural network toolbox}

The first number describes the major release. Version number V1.0 will be the first toolbox release which should have the same functions like the Matlab R14 SP3 neural network Toolbox.\\

The second number defines the finished functions. So to start, only the MLPs will realised and so this will be the number V0.1.0.\\

The third number defines the status of the actual development and function. V0.0.1 means no release with MLP, actually, everything under construction... ;-D.\\

Right now it's version V0.1.3 which means MLP works and currently the transfer function logsig is added.
\section{Code convention}

The main function of this toolbox will be programed with help of the book in \cite{4}.
So the variables will have the same names like in the book with one exception: Variables, only with one letter, will have two letters, e.g. 
\begin{itemize}
	\item $X \rightarrow Xx$
	\item $Q \rightarrow Qq$
	\item $a \rightarrow aa$
\end{itemize}
and so on ...\\

This is only to make it possible to search for variable names.

%\section{Varietes}
\label{chap:intro:sec:varietes}
\subsection{Object-oriented programming}
Some of the functions for Octave will have not the same name, like the matlab ones has.
This difference is because the object-oriented programming technology. The object-oriented functions will have a name postfix in Octave.\\
As example: \textit{isposint\_netw\_priv} means, \textit{isposint} is a private function from the matlab function \textit{network}.\\
This is a kind of "`simulation"' for the object-oriented programming technology.\\

\subsubsection{Data-Typ \textit{network} and m-file \textit{subsasgn}}
Matlab has a further data type \textit{network}. A basic neural network will be initialized in a
structure and than changed to the \textit{network} type with the \textit{class} command. The class command from Octave doesn't support the creation of this network type!
In the Matlab m-file \textit{network.m} on row 256, the network type will be created! From this moment, structure subscription assignment will call the file subsasgn from the \textit{@network} directory. Back in the \textit{newff} m-file on row 134, the @network subsasgn will be called the first time.

\paragraph{newff row 134; net.biasConnect} net=subsasgn(net,subscripts,v) will hold the following values:
\begin{itemize}
	\item net: the whole net structure
	\item subscripts: 1x1 structure;\\
		subscripts.type = '.'\\
		subscripts.subs = 'biasConnect'
	\item v: 2x1 double [1; 1]
\end{itemize}

\paragraph{newff row 135; net.inputConnect} net=subsasgn(net,subscripts,v) will hold the following values:
\begin{itemize}
	\item net: the whole net structure
	\item subscripts: 1x2 structure;\\
		subscripts(1).type = '.'\\
		subscripts(1).subs = 'inputConnect'\\
		subscripts(2).type = '()'\\
		subscripts(2).subs = [1] [1]
	\item v: 1x1 double 1
\end{itemize}
\textcolor{red}{bis hier sollte es erledigt sein...!}

\paragraph{newff row 137; net.layerConnect} net=subsasgn(net,subscripts,v) will hold the following values:
\begin{itemize}
	\item net: the whole net structure
	\item subscripts: 1x1 structure;\\
		subscripts(1).type = '.'\\
		subscripts(1).subs = 'layerConnect'
	\item v: 2x2 logical [0 0; 1 0]
\end{itemize}

\paragraph{newff row 138; net.outputConnect} net=subsasgn(net,subscripts,v) will hold the following values:
\begin{itemize}
	\item net: the whole net structure
	\item subscripts: 1x2 structure;\\
		subscripts(1).type = '.'\\
		subscripts(1).subs = 'outputConnect'\\
		subscripts(2).type = '()'\\
		subscripts(2).subs = '[2]'
	\item v: 1x1 double = 1
\end{itemize}

\paragraph{newff row 139; net.targetConnect} net=subsasgn(net,subscripts,v) will hold the following values:
\begin{itemize}
	\item net: the whole net structure
	\item subscripts: 1x2 structure;\\
		subscripts(1).type = '.'\\
		subscripts(1).subs = 'targetConnect'\\
		subscripts(2).type = '()'\\
		subscripts(2).subs = '[2]'
	\item v: 1x1 double = 1
\end{itemize}


\subsection{Data types}

\begin{table}
	\centering
	\begin{tabular}{c c c}
		\toprule
									& Matlab										& Octave\\
		\midrule
  								& double 											& double \\
									& int8  											& \\
									& int16 											& \\
									& int32 											& \\
									& int64 											& \\
			Numeric			& single											& \\
									& uint8 											& \\
									& uint16											& \\
									& uint32 											& \\ 
									& uint64 											& \\ 
									& complex 										& complex\\
		\midrule
			characters	& x														&  \\
		\midrule
			string			&															& \\
		\midrule
			cell				&	x														& x \\
		\midrule
			structure		& x 												  & x  \\
		\bottomrule
	\end{tabular}
\end{table}


%\chapter{Matlab}
This chapter describes the functions of the neuro toolbox by MathWorks.

\section{Matlab version}
\begin{itemize}
	\item MATLAB Version 7.1 (R14SP3)
	\item Neural Network Toolbox Version 4.0.6 (R14SP3)
\end{itemize}

%\section{Problems}
\subsection{Object-oriented programming}
The Neural Network Toolbox of Matlab is written in object-oriented programming style.
Octave can't handle this technology till 13. December 2005. In Matlab exist directorys with
the \textit{@}-Symbol in the name. Inside this directorys exist functions with the same name like
other functions. But this directory symbols the \textit{object-oriented programming style} and all
functions, called from the \textit{network.m}-file calls the functions inside the \textit{@network}-directory.
For this reason, it must be checked, which of the functions will be called at which position...\\
Good luck!

\subsection{isa}
The command \textit{isa} does not exist in Octave. So the behaviour of this function in Matlab will be testet and a work around for Octave developed.

\paragraph{isa(var,'double')}
Tabel \ref{tab:isaDouble} shows the properties of this command and the corresponding output for diffenrent input types.
\begin{table}
	\centering
	\begin{tabular}{c c}
		\toprule
																						& Matlab- \\
									 Input										& Output\\
		\midrule
  								 4 										  	  & 1 \\
									 4.0  										  &	1 \\
									 4.0+3i 										& 1 \\
									 char(4) 										& 0 \\
									 num2cell(4) 								& 0 \\
									 single(4)									& 0 \\
									 uint8(4) 									& 0\\
									 uint16(4)									&  0\\
									 uint32(4) 									& 0 \\ 
									 uint64(4) 									& 0 \\ 
									 complex(4) 								& 1 \\
									 a.a = 4 / Input: a.a  			& 1 \\
									 a.a = 4 / Input: a   			& 0 \\
		\bottomrule
	\end{tabular}
	\caption{Matlabs command \textbf{isa(a,'double')} and the corresponding output}
	\label{tab:isaDouble}
\end{table}

At this moment, there is no reason to work for a Octave version of \textbf{isa(a,'double')}, because I do not understand, why this variables must be of the type double!



 

\section{Matlab toolbox commands}
\subsection{Directory \textit{nncontrol}}

\paragraph{Directory \textit{nncontrol/private}}

\subsection{Directory \textit{nnet}}

\begin{itemize}
	\item adapt \textcolor{red}{\textit{not implemented}}
	\item boxdist \textcolor{red}{\textit{not implemented}}
	\item calca \textcolor{red}{\textit{not implemented}}
	\item calca1 \textcolor{red}{\textit{not implemented}}
	\item calce \textcolor{red}{\textit{not implemented}}
	\item calce1 \textcolor{red}{\textit{not implemented}}
	\item calcgx \textcolor{red}{\textit{not implemented}}
	\item calcjejj \textcolor{red}{\textit{not implemented}}
	\item calcjx  $\rightarrow$ is called calcjacobian
	\item calcpd \textcolor{red}{\textit{not implemented}}
	\item calcperf $\rightarrow$ implemented and also called calcperf
	\item combvec \textcolor{red}{\textit{not implemented}}
	\item compet \textcolor{red}{\textit{not implemented}}
	\item con2seq \textcolor{red}{\textit{not implemented}}
	\item concur \textcolor{red}{\textit{not implemented}}
	\item ddotprod \textcolor{red}{\textit{not implemented}}
	\item dhardlim \textcolor{red}{\textit{not implemented}}
	\item dhardlms \textcolor{red}{\textit{not implemented}}
	\item disp \textcolor{red}{\textit{not implemented}}
	\item display \textcolor{red}{\textit{not implemented}} 
	\item dist \textcolor{red}{\textit{not implemented}}
	\item dlogsig \textcolor{red}{\textit{not implemented}}
	\item dmae \textcolor{red}{\textit{not implemented}}
	\item dmse \textcolor{red}{\textit{not implemented}}
	\item dmsereg \textcolor{red}{\textit{not implemented}}
	\item dnetprod \textcolor{red}{\textit{not implemented}}
	\item dnetsum \textcolor{red}{\textit{not implemented}}
	\item dotprod \textcolor{red}{\textit{not implemented}}
	\item dposlin \textcolor{red}{\textit{not implemented}}
	\item dpurelin $\rightarrow$ implemented and also called dpurelin
	\item dradbas \textcolor{red}{\textit{not implemented}}
	\item dsatlin \textcolor{red}{\textit{not implemented}}
	\item dsatlins \textcolor{red}{\textit{not implemented}}
	\item dsse \textcolor{red}{\textit{not implemented}}
	\item dtansig $\rightarrow$ implemented and also called dtansig
	\item dtribas \textcolor{red}{\textit{not implemented}}
	\item errsurf \textcolor{red}{\textit{not implemented}}
	\item formx \textcolor{red}{\textit{not implemented}}
	\item gensim \textcolor{red}{\textit{not implemented}}
	\item getx $\rightarrow$ implemented and also called getx
	\item gridtop \textcolor{red}{\textit{not implemented}}
	\item hardlim \textcolor{red}{\textit{not implemented}}
	\item hardlims \textcolor{red}{\textit{not implemented}}
	\item hextop \textcolor{red}{\textit{not implemented}} 
	\item hintonw \textcolor{red}{\textit{not implemented}}
	\item hintonwb \textcolor{red}{\textit{not implemented}}
	\item ind2vec \textcolor{red}{\textit{not implemented}}
	\item init $\rightarrow$ implemented and also called init 
	\item initcon \textcolor{red}{\textit{not implemented}}
	\item initlay \textcolor{red}{\textit{not implemented}}
	\item initnw \textcolor{red}{\textit{not implemented}}
	\item initwb \textcolor{red}{\textit{not implemented}} 
	\item initzero \textcolor{red}{\textit{not implemented}}
	\item learncon \textcolor{red}{\textit{not implemented}}
	\item learngd \textcolor{red}{\textit{not implemented}}
	\item learngdm \textcolor{red}{\textit{not implemented}}
	\item learnh \textcolor{red}{\textit{not implemented}}
	\item learnhd \textcolor{red}{\textit{not implemented}}
	\item learnis \textcolor{red}{\textit{not implemented}}
	\item learnk \textcolor{red}{\textit{not implemented}}
	\item learnlv1 \textcolor{red}{\textit{not implemented}} 
	\item learnlv2 \textcolor{red}{\textit{not implemented}}
	\item learnos \textcolor{red}{\textit{not implemented}}
	\item learnp \textcolor{red}{\textit{not implemented}}
	\item learnpn \textcolor{red}{\textit{not implemented}}
	\item learnsom \textcolor{red}{\textit{not implemented}}
	\item learnwh \textcolor{red}{\textit{not implemented}}
	\item linkdist \textcolor{red}{\textit{not implemented}}
	\item logsig \textcolor{red}{\textit{not implemented}}
	\item mae \textcolor{red}{\textit{not implemented}}
	\item mandist \textcolor{red}{\textit{not implemented}}
	\item maxlinlr \textcolor{red}{\textit{not implemented}}
	\item midpoint \textcolor{red}{\textit{not implemented}}
	\item minmax $\rightarrow$ implemented and called min\_max
	\item mse $\rightarrow$ implemented and also called mse
	\item msereg \textcolor{red}{\textit{not implemented}}
	\item negdist \textcolor{red}{\textit{not implemented}}
	\item netprod \textcolor{red}{\textit{not implemented}}
	\item netsum \textcolor{red}{\textit{not implemented}}
	\item network $\rightarrow$ implemented and also called network
	\item newc \textcolor{red}{\textit{not implemented}}
	\item newcf \textcolor{red}{\textit{not implemented}}
	\item newelm \textcolor{red}{\textit{not implemented}}
	\item newff $\rightarrow$ implemented and also called newff
	\item newfftd \textcolor{red}{\textit{not implemented}}
	\item newgrnn \textcolor{red}{\textit{not implemented}}
	\item newhop \textcolor{red}{\textit{not implemented}}
	\item newlin \textcolor{red}{\textit{not implemented}}
	\item newlind \textcolor{red}{\textit{not implemented}}
	\item newlvq \textcolor{red}{\textit{not implemented}}
	\item newp \textcolor{red}{\textit{not implemented}}
	\item newpnn \textcolor{red}{\textit{not implemented}}
	\item newrb \textcolor{red}{\textit{not implemented}}
	\item newrbe \textcolor{red}{\textit{not implemented}}
	\item newsom \textcolor{red}{\textit{not implemented}}
	\item nncopy \textcolor{red}{\textit{not implemented}}
	\item nnt2c \textcolor{red}{\textit{not implemented}}
	\item nnt2elm \textcolor{red}{\textit{not implemented}}
	\item nnt2ff \textcolor{red}{\textit{not implemented}}
	\item nnt2hop \textcolor{red}{\textit{not implemented}}
	\item nnt2lin \textcolor{red}{\textit{not implemented}}
	\item nnt2lvq \textcolor{red}{\textit{not implemented}}
	\item nnt2p \textcolor{red}{\textit{not implemented}} 
	\item nnt2rb \textcolor{red}{\textit{not implemented}}
	\item nnt2som \textcolor{red}{\textit{not implemented}}
	\item nntool \textcolor{red}{\textit{not implemented}}
	\item normc \textcolor{red}{\textit{not implemented}}
	\item normprod \textcolor{red}{\textit{not implemented}}
	\item normr \textcolor{red}{\textit{not implemented}}
	\item plotbr \textcolor{red}{\textit{not implemented}}
	\item plotep \textcolor{red}{\textit{not implemented}}
	\item plotes \textcolor{red}{\textit{not implemented}}
	\item plotpc \textcolor{red}{\textit{not implemented}}
	\item plotperf \textcolor{red}{\textit{not implemented}}
	\item plotpv \textcolor{red}{\textit{not implemented}}
	\item plotsom \textcolor{red}{\textit{not implemented}}
	\item plotv \textcolor{red}{\textit{not implemented}}
	\item plotvec \textcolor{red}{\textit{not implemented}}
	\item pnormc \textcolor{red}{\textit{not implemented}}
	\item poslin \textcolor{red}{\textit{not implemented}}
	\item postmnmx \textcolor{red}{\textit{not implemented}}
	\item postreg \textcolor{red}{\textit{not implemented}}
	\item poststd \textcolor{red}{\textit{not implemented}}
	\item premnmx \textcolor{red}{\textit{not implemented}}
	\item prepca \textcolor{red}{\textit{not implemented}}
	\item prestd $\rightarrow$ implemented and also called prestd
	\item purelin $\rightarrow$ implemented and also called purelin
	\item quant \textcolor{red}{\textit{not implemented}}
	\item radbas \textcolor{red}{\textit{not implemented}}
	\item randnc \textcolor{red}{\textit{not implemented}}
	\item randnr \textcolor{red}{\textit{not implemented}}
	\item rands \textcolor{red}{\textit{not implemented}}
	\item randtop \textcolor{red}{\textit{not implemented}}
	\item revert \textcolor{red}{\textit{not implemented}}
	\item satlin \textcolor{red}{\textit{not implemented}}
	\item satlins \textcolor{red}{\textit{not implemented}}
	\item seq2con \textcolor{red}{\textit{not implemented}}
	\item setx $\rightarrow$ implemented and also called setx
	\item sim $\rightarrow$ implemented and also called sim
	\item softmax \textcolor{red}{\textit{not implemented}}
	\item srchbac \textcolor{red}{\textit{not implemented}}
	\item srchbre \textcolor{red}{\textit{not implemented}}
	\item srchcha \textcolor{red}{\textit{not implemented}}
	\item srchgol \textcolor{red}{\textit{not implemented}}
	\item srchhyb \textcolor{red}{\textit{not implemented}}
	\item sse \textcolor{red}{\textit{not implemented}}
	\item sumsqr \textcolor{red}{\textit{not implemented}}
	\item tansig $\rightarrow$ implemented and also called tansig
	\item train $\rightarrow$ implemented and also called train
	\item trainb \textcolor{red}{\textit{not implemented}}
	\item trainbfg \textcolor{red}{\textit{not implemented}}
	\item trainbr \textcolor{red}{\textit{not implemented}}
	\item trainc \textcolor{red}{\textit{not implemented}}
	\item traincgb \textcolor{red}{\textit{not implemented}}
	\item traincgf \textcolor{red}{\textit{not implemented}}
	\item traincgp \textcolor{red}{\textit{not implemented}}
	\item traingd \textcolor{red}{\textit{not implemented}}
	\item traingda \textcolor{red}{\textit{not implemented}}
	\item traingdm \textcolor{red}{\textit{not implemented}}
	\item traingdx \textcolor{red}{\textit{not implemented}} 
	\item trainlm $\rightarrow$ implemented and also called trainlm
	\item trainoss \textcolor{red}{\textit{not implemented}}
	\item trainr \textcolor{red}{\textit{not implemented}}
	\item trainrp \textcolor{red}{\textit{not implemented}}
	\item trains \textcolor{red}{\textit{not implemented}}
	\item trainscg \textcolor{red}{\textit{not implemented}}
	\item tramnmx \textcolor{red}{\textit{not implemented}}
	\item trapca \textcolor{red}{\textit{not implemented}}
	\item trastd $\rightarrow$ implemented and also called trastd
	\item tribas \textcolor{red}{\textit{not implemented}}
	\item vec2ind \textcolor{red}{\textit{not implemented}}
\end{itemize} 
\chapter{Matlab}
This chapter describes the functions of the neuro toolbox by MathWorks.

\section{Matlab version}
\begin{itemize}
	\item MATLAB Version 7.1 (R14SP3)
	\item Neural Network Toolbox Version 4.0.6 (R14SP3)
\end{itemize}

%\section{Problems}
\subsection{Object-oriented programming}
The Neural Network Toolbox of Matlab is written in object-oriented programming style.
Octave can't handle this technology till 13. December 2005. In Matlab exist directorys with
the \textit{@}-Symbol in the name. Inside this directorys exist functions with the same name like
other functions. But this directory symbols the \textit{object-oriented programming style} and all
functions, called from the \textit{network.m}-file calls the functions inside the \textit{@network}-directory.
For this reason, it must be checked, which of the functions will be called at which position...\\
Good luck!

\subsection{isa}
The command \textit{isa} does not exist in Octave. So the behaviour of this function in Matlab will be testet and a work around for Octave developed.

\paragraph{isa(var,'double')}
Tabel \ref{tab:isaDouble} shows the properties of this command and the corresponding output for diffenrent input types.
\begin{table}
	\centering
	\begin{tabular}{c c}
		\toprule
																						& Matlab- \\
									 Input										& Output\\
		\midrule
  								 4 										  	  & 1 \\
									 4.0  										  &	1 \\
									 4.0+3i 										& 1 \\
									 char(4) 										& 0 \\
									 num2cell(4) 								& 0 \\
									 single(4)									& 0 \\
									 uint8(4) 									& 0\\
									 uint16(4)									&  0\\
									 uint32(4) 									& 0 \\ 
									 uint64(4) 									& 0 \\ 
									 complex(4) 								& 1 \\
									 a.a = 4 / Input: a.a  			& 1 \\
									 a.a = 4 / Input: a   			& 0 \\
		\bottomrule
	\end{tabular}
	\caption{Matlabs command \textbf{isa(a,'double')} and the corresponding output}
	\label{tab:isaDouble}
\end{table}

At this moment, there is no reason to work for a Octave version of \textbf{isa(a,'double')}, because I do not understand, why this variables must be of the type double!



 

\section{Matlab toolbox commands}
\subsection{Directory \textit{nncontrol}}

\paragraph{Directory \textit{nncontrol/private}}

\subsection{Directory \textit{nnet}}

\begin{itemize}
	\item adapt \textcolor{red}{\textit{not implemented}}
	\item boxdist \textcolor{red}{\textit{not implemented}}
	\item calca \textcolor{red}{\textit{not implemented}}
	\item calca1 \textcolor{red}{\textit{not implemented}}
	\item calce \textcolor{red}{\textit{not implemented}}
	\item calce1 \textcolor{red}{\textit{not implemented}}
	\item calcgx \textcolor{red}{\textit{not implemented}}
	\item calcjejj \textcolor{red}{\textit{not implemented}}
	\item calcjx  $\rightarrow$ is called calcjacobian
	\item calcpd \textcolor{red}{\textit{not implemented}}
	\item calcperf $\rightarrow$ implemented and also called calcperf
	\item combvec \textcolor{red}{\textit{not implemented}}
	\item compet \textcolor{red}{\textit{not implemented}}
	\item con2seq \textcolor{red}{\textit{not implemented}}
	\item concur \textcolor{red}{\textit{not implemented}}
	\item ddotprod \textcolor{red}{\textit{not implemented}}
	\item dhardlim \textcolor{red}{\textit{not implemented}}
	\item dhardlms \textcolor{red}{\textit{not implemented}}
	\item disp \textcolor{red}{\textit{not implemented}}
	\item display \textcolor{red}{\textit{not implemented}} 
	\item dist \textcolor{red}{\textit{not implemented}}
	\item dlogsig \textcolor{red}{\textit{not implemented}}
	\item dmae \textcolor{red}{\textit{not implemented}}
	\item dmse \textcolor{red}{\textit{not implemented}}
	\item dmsereg \textcolor{red}{\textit{not implemented}}
	\item dnetprod \textcolor{red}{\textit{not implemented}}
	\item dnetsum \textcolor{red}{\textit{not implemented}}
	\item dotprod \textcolor{red}{\textit{not implemented}}
	\item dposlin \textcolor{red}{\textit{not implemented}}
	\item dpurelin $\rightarrow$ implemented and also called dpurelin
	\item dradbas \textcolor{red}{\textit{not implemented}}
	\item dsatlin \textcolor{red}{\textit{not implemented}}
	\item dsatlins \textcolor{red}{\textit{not implemented}}
	\item dsse \textcolor{red}{\textit{not implemented}}
	\item dtansig $\rightarrow$ implemented and also called dtansig
	\item dtribas \textcolor{red}{\textit{not implemented}}
	\item errsurf \textcolor{red}{\textit{not implemented}}
	\item formx \textcolor{red}{\textit{not implemented}}
	\item gensim \textcolor{red}{\textit{not implemented}}
	\item getx $\rightarrow$ implemented and also called getx
	\item gridtop \textcolor{red}{\textit{not implemented}}
	\item hardlim \textcolor{red}{\textit{not implemented}}
	\item hardlims \textcolor{red}{\textit{not implemented}}
	\item hextop \textcolor{red}{\textit{not implemented}} 
	\item hintonw \textcolor{red}{\textit{not implemented}}
	\item hintonwb \textcolor{red}{\textit{not implemented}}
	\item ind2vec \textcolor{red}{\textit{not implemented}}
	\item init $\rightarrow$ implemented and also called init 
	\item initcon \textcolor{red}{\textit{not implemented}}
	\item initlay \textcolor{red}{\textit{not implemented}}
	\item initnw \textcolor{red}{\textit{not implemented}}
	\item initwb \textcolor{red}{\textit{not implemented}} 
	\item initzero \textcolor{red}{\textit{not implemented}}
	\item learncon \textcolor{red}{\textit{not implemented}}
	\item learngd \textcolor{red}{\textit{not implemented}}
	\item learngdm \textcolor{red}{\textit{not implemented}}
	\item learnh \textcolor{red}{\textit{not implemented}}
	\item learnhd \textcolor{red}{\textit{not implemented}}
	\item learnis \textcolor{red}{\textit{not implemented}}
	\item learnk \textcolor{red}{\textit{not implemented}}
	\item learnlv1 \textcolor{red}{\textit{not implemented}} 
	\item learnlv2 \textcolor{red}{\textit{not implemented}}
	\item learnos \textcolor{red}{\textit{not implemented}}
	\item learnp \textcolor{red}{\textit{not implemented}}
	\item learnpn \textcolor{red}{\textit{not implemented}}
	\item learnsom \textcolor{red}{\textit{not implemented}}
	\item learnwh \textcolor{red}{\textit{not implemented}}
	\item linkdist \textcolor{red}{\textit{not implemented}}
	\item logsig \textcolor{red}{\textit{not implemented}}
	\item mae \textcolor{red}{\textit{not implemented}}
	\item mandist \textcolor{red}{\textit{not implemented}}
	\item maxlinlr \textcolor{red}{\textit{not implemented}}
	\item midpoint \textcolor{red}{\textit{not implemented}}
	\item minmax $\rightarrow$ implemented and called min\_max
	\item mse $\rightarrow$ implemented and also called mse
	\item msereg \textcolor{red}{\textit{not implemented}}
	\item negdist \textcolor{red}{\textit{not implemented}}
	\item netprod \textcolor{red}{\textit{not implemented}}
	\item netsum \textcolor{red}{\textit{not implemented}}
	\item network $\rightarrow$ implemented and also called network
	\item newc \textcolor{red}{\textit{not implemented}}
	\item newcf \textcolor{red}{\textit{not implemented}}
	\item newelm \textcolor{red}{\textit{not implemented}}
	\item newff $\rightarrow$ implemented and also called newff
	\item newfftd \textcolor{red}{\textit{not implemented}}
	\item newgrnn \textcolor{red}{\textit{not implemented}}
	\item newhop \textcolor{red}{\textit{not implemented}}
	\item newlin \textcolor{red}{\textit{not implemented}}
	\item newlind \textcolor{red}{\textit{not implemented}}
	\item newlvq \textcolor{red}{\textit{not implemented}}
	\item newp \textcolor{red}{\textit{not implemented}}
	\item newpnn \textcolor{red}{\textit{not implemented}}
	\item newrb \textcolor{red}{\textit{not implemented}}
	\item newrbe \textcolor{red}{\textit{not implemented}}
	\item newsom \textcolor{red}{\textit{not implemented}}
	\item nncopy \textcolor{red}{\textit{not implemented}}
	\item nnt2c \textcolor{red}{\textit{not implemented}}
	\item nnt2elm \textcolor{red}{\textit{not implemented}}
	\item nnt2ff \textcolor{red}{\textit{not implemented}}
	\item nnt2hop \textcolor{red}{\textit{not implemented}}
	\item nnt2lin \textcolor{red}{\textit{not implemented}}
	\item nnt2lvq \textcolor{red}{\textit{not implemented}}
	\item nnt2p \textcolor{red}{\textit{not implemented}} 
	\item nnt2rb \textcolor{red}{\textit{not implemented}}
	\item nnt2som \textcolor{red}{\textit{not implemented}}
	\item nntool \textcolor{red}{\textit{not implemented}}
	\item normc \textcolor{red}{\textit{not implemented}}
	\item normprod \textcolor{red}{\textit{not implemented}}
	\item normr \textcolor{red}{\textit{not implemented}}
	\item plotbr \textcolor{red}{\textit{not implemented}}
	\item plotep \textcolor{red}{\textit{not implemented}}
	\item plotes \textcolor{red}{\textit{not implemented}}
	\item plotpc \textcolor{red}{\textit{not implemented}}
	\item plotperf \textcolor{red}{\textit{not implemented}}
	\item plotpv \textcolor{red}{\textit{not implemented}}
	\item plotsom \textcolor{red}{\textit{not implemented}}
	\item plotv \textcolor{red}{\textit{not implemented}}
	\item plotvec \textcolor{red}{\textit{not implemented}}
	\item pnormc \textcolor{red}{\textit{not implemented}}
	\item poslin \textcolor{red}{\textit{not implemented}}
	\item postmnmx \textcolor{red}{\textit{not implemented}}
	\item postreg \textcolor{red}{\textit{not implemented}}
	\item poststd \textcolor{red}{\textit{not implemented}}
	\item premnmx \textcolor{red}{\textit{not implemented}}
	\item prepca \textcolor{red}{\textit{not implemented}}
	\item prestd $\rightarrow$ implemented and also called prestd
	\item purelin $\rightarrow$ implemented and also called purelin
	\item quant \textcolor{red}{\textit{not implemented}}
	\item radbas \textcolor{red}{\textit{not implemented}}
	\item randnc \textcolor{red}{\textit{not implemented}}
	\item randnr \textcolor{red}{\textit{not implemented}}
	\item rands \textcolor{red}{\textit{not implemented}}
	\item randtop \textcolor{red}{\textit{not implemented}}
	\item revert \textcolor{red}{\textit{not implemented}}
	\item satlin \textcolor{red}{\textit{not implemented}}
	\item satlins \textcolor{red}{\textit{not implemented}}
	\item seq2con \textcolor{red}{\textit{not implemented}}
	\item setx $\rightarrow$ implemented and also called setx
	\item sim $\rightarrow$ implemented and also called sim
	\item softmax \textcolor{red}{\textit{not implemented}}
	\item srchbac \textcolor{red}{\textit{not implemented}}
	\item srchbre \textcolor{red}{\textit{not implemented}}
	\item srchcha \textcolor{red}{\textit{not implemented}}
	\item srchgol \textcolor{red}{\textit{not implemented}}
	\item srchhyb \textcolor{red}{\textit{not implemented}}
	\item sse \textcolor{red}{\textit{not implemented}}
	\item sumsqr \textcolor{red}{\textit{not implemented}}
	\item tansig $\rightarrow$ implemented and also called tansig
	\item train $\rightarrow$ implemented and also called train
	\item trainb \textcolor{red}{\textit{not implemented}}
	\item trainbfg \textcolor{red}{\textit{not implemented}}
	\item trainbr \textcolor{red}{\textit{not implemented}}
	\item trainc \textcolor{red}{\textit{not implemented}}
	\item traincgb \textcolor{red}{\textit{not implemented}}
	\item traincgf \textcolor{red}{\textit{not implemented}}
	\item traincgp \textcolor{red}{\textit{not implemented}}
	\item traingd \textcolor{red}{\textit{not implemented}}
	\item traingda \textcolor{red}{\textit{not implemented}}
	\item traingdm \textcolor{red}{\textit{not implemented}}
	\item traingdx \textcolor{red}{\textit{not implemented}} 
	\item trainlm $\rightarrow$ implemented and also called trainlm
	\item trainoss \textcolor{red}{\textit{not implemented}}
	\item trainr \textcolor{red}{\textit{not implemented}}
	\item trainrp \textcolor{red}{\textit{not implemented}}
	\item trains \textcolor{red}{\textit{not implemented}}
	\item trainscg \textcolor{red}{\textit{not implemented}}
	\item tramnmx \textcolor{red}{\textit{not implemented}}
	\item trapca \textcolor{red}{\textit{not implemented}}
	\item trastd $\rightarrow$ implemented and also called trastd
	\item tribas \textcolor{red}{\textit{not implemented}}
	\item vec2ind \textcolor{red}{\textit{not implemented}}
\end{itemize} 
\chapter{Examples}


\section{Example 1}

This MLP is designed with 2-2-1. This is not a complete example but it will help to understand the
dimensions of all matrices and vectores are used inside the Levenberg-Marquardt algorithm.

\subsection{Data matrices}
The input matrix will be defined like in equation \eqref{equ:mInput} and the output matrix like in 
equation \eqref{equ:mOutput}.

\begin{equation}
  mInput = \left[ \begin{array}{c c c c}
												1 & 2 & 3 & 1   \\
												1	& 1 & 1 & 2 	\\
												1 & 2 & 1 & 2		\\																					
												\end{array}
					\right]
		\label{equ:mInput}
\end{equation}

\begin{equation}
  mOutput = \left[ \begin{array}{c c c c}
												1 & 1.5 & 2 & 3   \\																					
									 \end{array}
					\right]
		\label{equ:mOutput}
\end{equation}




\subsection{Weight matrices}
The first layer matrix will hold 2x3 weights. The second layer matrix will hold 1x2 weights.
The first bias holds 3x1 weights and the second holds only a scalar element.

\subsection{Sensitivity Matrices}
This part is right now not so clear in my mind. What is the dimension of these two matrices?
The first layer sensitivity matrix should be about 2x71. Number of hidden neurons in the rows and number of train data sets in the columns.\\

In the actual version, the dimension is about 71x71 .. so it seems to have a mistake inside the algorithm :-(


%\chapter{Test}
This chapter describes how the functions are tested.



\section{isposint\_netw\_priv}
\subsubsection{Short description}
Checks if value is a positive integer.\\

\subsubsection{Matlab}
It is the same name as in Matlab but not the totaly same code. The Matlab version is in the directory \textit{nnet/@network/private}. This means, it is a private function of the class \textit{network}! \linebreak

\subsubsection{Test description}
The test file is called \textit{isposint\_test.m} \textcolor{red}{localized at ...}\\
There are only five tests:\\
negative integer: $-5$ result should be 0\\
negative float: $-5.3$ result should be 0\\
positive integer: $0$ result should be 1\\
positive integer: $5$ result should be 1\\
positive float: $5.3$ result should be 0\\
This tests run seccessfully!


\subsection{min\_max}
\textit{min\_max} get the minimal and maximal values of an training input matrix. So the dimension of this matrix must be an RxN matrix where R is the number of input neurons and N depends on the number of training sets.\\

\noindent \textbf{\textcolor{brown}{Syntax:}}\\

\noindent mMinMaxElements = min\_max(RxN);\\

\noindent \textbf{\textcolor{brown}{Description:}}\\

\noindent RxN: R x N matrix of min and max values for R input elements with N columns\\ 

\noindent \textbf{\textcolor{brown}{Example:}}\\

\begin{equation}
	\left[
		\begin{array}{cc}
     	1 &  11 \\
    	0  & 21
   \end{array} 
	 \right]            = min\_max\left[ 
	 															   \begin{array}{ccccc}
	 															   3 & 1 & 3 & 5 & 11 \\
	 															   12& 0 & 21& 8 & 6  \\
	 															   \end{array}
	 															   	 \right]
\end{equation}



% Preamble

%\documentclass[a4paper]{report}

%\usepackage[ngerman]{babel}
%\usepackage[T1]{fontenc}
%\usepackage[ansinew]{inputenc}


%%%%%%%%%%%%%%%%%%%%%%%%%%%%%%
% start text here!!

%\begin{document}

\begin{thebibliography}{XXXXXXX}

\bibitem [1]{1} John W. Eaton

GNU Octave Manual, Edition 3, PDF-Version, February 1997

\bibitem [2]{2} The MathWorks, Inc.

MATLAB Online-Help

\bibitem [3]{3} Steven W. Smith

The Scientist and Engineer's Guide to Digital Signal Processing
ISBN 0-9660176-3-3, California Technical Publishing, 1997

\bibitem [4]{4} Martin T. Hagan, Howard B. Demuth, Mark Beale

Neural Network Design, ISBN 0971732108, PWS Publishing Company, USA, Boston, 1996





\end{thebibliography}
%\end{document}



\end{document}


