%% LyX 1.3 created this file.  For more info, see http://www.lyx.org/.
%% Do not edit unless you really know what you are doing.
\documentclass[11pt,english]{article}
\usepackage{palatino}
\usepackage[T1]{fontenc}
\usepackage{a4}
\usepackage{verbatim}
\usepackage{amsmath}
\usepackage{setspace}
\onehalfspacing
\usepackage{amssymb}
\IfFileExists{url.sty}{\usepackage{url}}
                      {\newcommand{\url}{\texttt}}

\makeatletter

%%%%%%%%%%%%%%%%%%%%%%%%%%%%%% LyX specific LaTeX commands.
%% Because html converters don't know tabularnewline
\providecommand{\tabularnewline}{\\}

%%%%%%%%%%%%%%%%%%%%%%%%%%%%%% User specified LaTeX commands.
%%%%%%%%%%%%%%%%%%%%%%%%%%%%%%%%%%%%%%%%%%%%%%%%%%%%%%%%%%%%%%%%%%%%%%%%%%%%%%%%%%%%%%%%%%%%%%%%%%%%%%%%%%%%%%%%%%%%%%%%%%%%
\newtheorem{theorem}{Theorem}
\newtheorem{acknowledgement}[theorem]{Acknowledgement}
\newtheorem{axiom}[theorem]{Axiom}
\newtheorem{case}[theorem]{Case}
\newtheorem{claim}[theorem]{Claim}
\newtheorem{conclusion}[theorem]{Conclusion}
\newtheorem{condition}[theorem]{Condition}
\newtheorem{conjecture}[theorem]{Conjecture}
\newtheorem{corollary}[theorem]{Corollary}
\newtheorem{criterion}[theorem]{Criterion}
\newtheorem{definition}[theorem]{Definition}
\newtheorem{example}[theorem]{Example}
\newtheorem{exercise}[theorem]{Exercise}
\newtheorem{lemma}[theorem]{Lemma}
\newtheorem{notation}[theorem]{Notation}
\newtheorem{problem}[theorem]{Problem}
\newtheorem{proposition}[theorem]{Proposition}
\newtheorem{remark}[theorem]{Remark}
\newtheorem{solution}[theorem]{Solution}
\newtheorem{summary}[theorem]{Summary}
\usepackage{graphicx}
\usepackage{amsmath}
\usepackage{html}
\usepackage{dcolumn}
\usepackage{mathpazo}
\usepackage[ps2pdf,pdftitle={Econometrics},urlcolor=blue,linktocpage,a4paper,colorlinks=true]{hyperref}

\usepackage{babel}
\makeatother
\begin{document}

\title{Unconstrained Optimization with MINTOOLKIT for GNU Octave}


\author{Michael Creel%
\footnote{Dept. of Economics and Economic History, Universitat Aut�noma de Barcelona.
michael.creel@uab.es%
}}


\date{March 2004}

\maketitle
\begin{abstract}
The paper documents MINTOOLKIT for GNU Octave. MINTOOLKIT provides
functions for minimization and numeric differentiation. The main algorithms
are BFGS, LBFGS, and simulated annealing. Examples are given.
\end{abstract}

\section{Introduction}

Unconstrained optimization is a basic tool in many disciplines, and
there is no need to discuss its importance. This paper discusses how
unconstrained optimization may be done using GNU Octave (Eaton, \htmladdnormallink{www.octave.org}{http://www.octave.org})
using the package MINTOOLKIT (Creel, \htmladdnormallink{http://pareto.uab.es/mcreel/MINTOOLKIT}{http://pareto.uab.es/mcreel/MINTOOLKIT}).
If you would just like to see some examples of how to use the algorithms,
skip to section \ref{sec:Examples}. Otherwise, here's some introductory
information that explains how algorithms we selected for inclusion
into MINTOOLKIT.


\subsection{Types of problems}

We first briefly discuss types of optimization problems, in order
to identify the cases where GNU Octave will be a good platform for
analysis, and which of the algorithms in MINTOOLKIT will likely work
well for given cases.


\subsubsection{Large/small}

What is the number of parameters to be minimized? Let $k$ be the
number of parameters. Memory usage of an algorithm will be some function
$f(k).$ If this function is growing rapidly for a certain algorithm,
that algorithm will cease to be useful for ''large'' problems, since
memory will be exhausted. Of course, ''large'' is a relative term
that increases over time as memory becomes cheaper. For ''small''
problems, the speed of convergence of the algorithm will be of primary
importance, since memory resources will not be a bottleneck.


\subsubsection{Continuous/discontinuous}

Gradient-based methods such as Newton's algorithm or quasi-Newton
methods rely on the function being differentiable. This will not hold
if the function is not continuous. Search-type algorithms will be
appropriate here.


\subsubsection{Convex/nonconvex}

A convex objective function will have a single global minimizer, whereas
nonconvex functions may have additional local minima. Quasi-Newton
methods only use local information in their updates, so they may well
converge to a non-global minimum, depending upon starting values.
A possible solution is to try a number of starting values. This is
likely to work well if the nonconvexity problem is not too severe.
When there are \emph{many} local minima, a search-type algorithm may
become more efficient, since the problem of local minima is dealt
with automatically and doesn't require the analysts' intervention.
After all, who's time is more important, yours, or your computer's?


\subsubsection{Costly/cheap}

Can the objective function be evaluated quickly, or is it time-consuming?
Octave is an interpreted language, and is in general slower than FORTRAN,
C, or similar. So expensive objective functions are best not implemented
in pure Octave. One might go to a different environment for analysis,
but in fact it is relatively easy to convert objective functions written
in Octave to C++, and call them dynamically from Octave scripts. In
this way, the expensive calculations are done using a fast language,
while the user deals with the convenient, friendly Octave environment.
In fact, C++ functions may make use of the Octave classes, so converting
an Octave function to C++ is not very difficult. The algorithms in
MINTOOLKIT serve as examples of how this may be done.


\section{Algorithms}

The algorithms in MINTOOLKIT were chosen based upon many sources of
information, two of which are \htmladdnormallink{Nocedal (1992)}{http://www.ece.northwestern.edu/~nocedal/PDFfiles/acta.pdf},
for continuous, convex problems, and \htmladdnormallink{Mittelmann}{http://plato.la.asu.edu/topics/problems/global.html}
for global minimization. The goal of MINTOOLKIT is to be able to solve
well-posed problems quickly and robustly, using the smallest set of
algorithms possible. ''Well-posed'' is an important adjective here
- MINTOOLKIT is not meant to be able to solve poorly conditioned problems
or problems that can easily fall into numeric precision traps. Please
pay attention to how your data is scaled, try to ''bullet-proof''
your objective function to avoid divisions by zero, etc. Nevertheless,
if you find bugs, or have suggestions or comments, please contact
the author. 


\subsection{BFGSMIN}

The BFGS algorithm is probably the most widely-used quasi-Newton method
for moderately-sized continuous problems that are not extremely nonconvex.
It is more robust than other quasi-Newton methods such as DFP (Nocedal,
1992), and it is faster than Newton's method, since the Hessian matrix
need not be calculated. One could easily create a Newton algorithm
using the source code for \texttt{bfgsmin}.


\subsection{LBFGSMIN}

For large problems, the BFGS algorithm may not be feasible, since
it requires storing a $k\times k$ matrix. The LBFGS (''L'' is for
''limited memory'') method is able to store all information for
updates in vectors, which substantially reduced memory requirements.
It may also be faster than the BFGS algorithm in some circumstances,
since it may use fewer floating point operations per iteration, depending
upon the size of the problem. While it will usually require more iterations
that BFGS, it may be faster if each iteration is faster. \htmladdnormallink{Liu and Nocecal \(1989\)}{http://www.ece.northwestern.edu/~nocedal/PDFfiles/limited-memory.pdf}
is a reference.


\subsection{SAMIN}

When problems are mildly nonconvex, the quasi-Newton methods above,
combined with a number of trial start values, may be the fastest way
to find the global minimum. This solution is implemented in \texttt{battery.m.}
But when the problem becomes less smooth, with many local minima,
this solution may fail. Simulated annealing is one of the algorithms
that works well in this case. The implementation by \htmladdnormallink{Goffe}{http://www.netlib.no/netlib/opt/simann.f}
has been used widely, and is the basis for the version included in
MINTOOLKIT. An additional reference is Goffe (1996).

\texttt{samin} differs from the Goffe code in two important ways:

\begin{enumerate}
\item The ''temperature'' is determined automatically. In a first stage,
the temperature is increased until the active search region covers
the entire parameter space defined as the $k$-dimensional rectangle
$\times_{j=1}^{k}(lb_{j},ub_{j})$ where $lb_{j}$ and $ub_{j}$ are
the $j$th elements of the LB and UB vectors that are the lower and
upper bounds for the parameters (these are user-specified arguments
to samin. Once this is achieved, the temperature decreases as usual.
\item Convergence is defined as \emph{two} conditions holding simultaneously.

\begin{enumerate}
\item The last NEPS best function values cannot differ by more than FUNCTOL.
This is as in Goffe's code.
\item The width of the search interval must be less than PARAMTOL for each
parameter. This allows to avoid accepting points on a flat plateau.
\end{enumerate}
\end{enumerate}

\section{Obtaining the code}

\begin{itemize}
\item MINTOOLKIT is available directly from the author, at \htmladdnormallink{http://pareto.uab.es/mcreel/MINTOOLKIT}{http://pareto.uab.es/mcreel/MINTOOLKIT}.
If you get it this way, uncompress the file where you like, change
to the MINTOOLKIT directory, and compile by typing ''make all''
(this supposes that you have Octave installed already). Make sure
that Octave knows where the MINTOOLKIT directory is. This option guarantees
that you have the most recent version.
\item Otherwise, you can obtain MINTOOLKIT as part of the \htmladdnormallink{octave-forge}{http://sourceforge.net/project/showfiles.php?group_id=2888}
package.
\item If you happen to be running \htmladdnormallink{Debian}{http://www.debian.org}
\htmladdnormallink{Linux}{http://www.linux.org}, you can install
a pre-compiled version of octave-forge and all required files by typing
''apt-get install octave-forge''. This is the easiest (and recommended)
option.
\end{itemize}

\section{\label{sec:Examples}Examples}

MINTOOLKIT contains some functions for use by users, and some other
functions that users can ignore. The functions for users are
\medskip{}

\begin{tabular}{|c|c|}
\hline 
Function&
Purpose\tabularnewline
\hline
\hline 
bfgsmin&
Ordinary BFGS algorithm\tabularnewline
\hline 
battery&
Calls bfgsmin with a set of starting values\tabularnewline
\hline 
lbfgsmin&
Limited-memory BFGS, for large problems\tabularnewline
\hline 
samin&
Simulated annealing, for global minimization\tabularnewline
\hline 
numgradient&
numeric first derivative of vector-valued function\tabularnewline
\hline 
numhessian&
numeric second derivative matrix\tabularnewline
\hline
\end{tabular}
\medskip{}

This section gives some very simple examples of the use of the algorithms
and functions in MINTOOLKIT. The first examples are intended to clearly
illustrate how to use the algorithms. Realism is not important. Then
some more difficult problems are considered.

The functions in MINTOOLKIT allow minimization or differentiation
with respect to any of the arguments of a function, holding the other
arguments fixed. The other arguments can include data or fixed parameters
of the function, for example. The argument with respect to which minimization
or differentiation is done is denoted by $minarg$, which by default
is equal to 1. Any function to be minimized or differentiated by algorithms
in MINTOOLKIT must follow one of the forms

\begin{eqnarray*}
value & = & \, f(arg_{1},\, arg_{2},\,...,\, arg_{p})\\
{}[value,\, return_{2},\,...,\, return_{n}] & = & \, f(arg_{1},\, arg_{2},\,...,\, arg_{p})\end{eqnarray*}
\emph{Special case}: If the second form is used \emph{and} $return_{2}$
is a $k\times1$ vector, where $k$ is the dimension of $minarg$,
then is assumed to be the gradient of $f$ with respect to $minarg$,
if the algorithm called uses the gradient. Otherwise, it (and any
other returns from $f$) are ignored by MINTOOLKIT.


\subsection{Minimization}


\subsubsection{\texttt{bfgsmin}}

\texttt{bfgsmin} is called as \[
[theta,\, value,\, convergence]=bfgsmin("f",\,\{ args\},\,\{ control\})\]
The first argument $"f"$ is a string variable that holds the name
of the function to be minimized. The second argument, $args$, is
a cell array that hold the arguments of $f$. The third argument $control$
is an optional cell array of $4$ elements. The elements of $control$
are described in Table \ref{cap:Controls-for-bfgsmin}. %
\begin{table}

\caption{\label{cap:Controls-for-bfgsmin}Controls for \texttt{bfgsmin}}

\begin{tabular}{|c|c|c|c|}
\hline 
{\footnotesize Element}&
{\footnotesize Purpose}&
{\footnotesize Default Value}&
{\footnotesize Other possible values}\tabularnewline
\hline
\hline 
{\footnotesize 1}&
{\footnotesize maxiters}&
{\footnotesize -1 (infinity)}&
{\footnotesize any positive integer}\tabularnewline
\hline 
{\footnotesize 2}&
{\footnotesize verbosity}&
{\footnotesize 0}&
{\footnotesize 1: summary every iteration; 2: only final summary}\tabularnewline
\hline 
{\footnotesize 3}&
{\footnotesize criterion}&
{\footnotesize 1: strict convergence (f, g, $\Delta p$)}&
{\footnotesize 2: weak convergence (only f)}\tabularnewline
\hline 
{\footnotesize 4}&
{\footnotesize minarg}&
{\footnotesize 1: first argument}&
{\footnotesize int: $1\le minarg\le k$, $k=\# args$}\tabularnewline
\hline
\end{tabular}
\end{table}
The outputs of \texttt{bfgsmin} are obvious, except the code values
that $convergence$ can take on. These are -1 for no convergence,
maxiters exceeded; 1: convergence according to the specified strong
or weak criterion; 2: no convergence due to failure of the algorithm
(\emph{e.g.,} the gradient calculation fails, or a stepsize cannot
be found).

Consider a simple example - minimizing a quadratic function. The program
\htmladdnormallink{bfgsmin-example.m}{http://pareto.uab.es/mcreel/MINTOOLKIT/bfgsmin-example.m}
follows:\verbatiminput{bfgsmin-example.m}

\begin{itemize}
\item The first example uses numeric derivatives, and minimizes with respect
to $x,$ the first argument of the objective function. The second
argument, $y,$ is treated as fixed.
\item The second example uses analytic derivatives, since it calls objective2,
and minimizes with respect to $x,$ the first argument of the objective
function. The second argument, $y,$ is treated as fixed.
\item The third example uses numeric derivatives, and minimizes with respect
to $y,$ the second argument of the objective function, since the
4th element of \texttt{control}, \texttt{minarg}, is 2 . The first
argument, $x,$ is treated as fixed.
\end{itemize}
The output of running this example is

\verbatiminput{bfsg-example.out}Notice that analytic gradients lead
to faster convergence that do numeric gradients. Also note in the
third example, where \texttt{minarg=2}, that minimization can be with
respect to any of the arguments of the objective function.


\subsubsection{\texttt{lbfgsmin}}

When the problem is very large, a limited-memory bfgs algorithm may
be needed, if \texttt{bfgsmin} is not feasible due to memory limitations.
\texttt{lbfgsmin} is called as \[
[theta,\, value,\, convergence]=lbfgsmin("f",\,\{ args\},\,\{ control\})\]
The first argument $"f"$ is a string variable that holds the name
of the function to be minimized. The second argument, $args$, is
a cell array that hold the arguments of $f$. The third argument $control$
is an optional cell array of $5$ elements. The elements of $control$
are the same as for \texttt{bfgsmin,} except that there is one more
element that controls how many iterations are used to form the quasi-Hessian
matrix (this is the \emph{memory} of the method). The control vector
is fully described in Table \ref{cap:Controls-for-lbfgsmin}. You
can easily modify the above example to use the \texttt{lbfgsmin} method.
%
\begin{table}

\caption{\label{cap:Controls-for-lbfgsmin}Controls for \texttt{lbfgsmin}}

\begin{tabular}{|c|c|c|c|}
\hline 
{\footnotesize Element}&
{\footnotesize Purpose}&
{\footnotesize Default Value}&
{\footnotesize Other possible values}\tabularnewline
\hline
\hline 
{\footnotesize 1}&
{\footnotesize maxiters}&
{\footnotesize -1 (infinity)}&
{\footnotesize any positive integer}\tabularnewline
\hline 
{\footnotesize 2}&
{\footnotesize verbosity}&
{\footnotesize 0}&
{\footnotesize 1: summary every iteration; 2: only final summary}\tabularnewline
\hline 
{\footnotesize 3}&
{\footnotesize criterion}&
{\footnotesize 1: strict convergence (f, g, $\Delta p$)}&
{\footnotesize 2: weak convergence (only f)}\tabularnewline
\hline 
{\footnotesize 4}&
{\footnotesize minarg}&
{\footnotesize 1: first argument}&
{\footnotesize int: $1\le minarg\le k$, $k=\# args$}\tabularnewline
\hline 
{\footnotesize 5}&
{\footnotesize memory}&
{\footnotesize 5}&
{\footnotesize any positive integer}\tabularnewline
\hline
\end{tabular}
\end{table}


It is possible that \texttt{lbfgsmin} can outperform \texttt{bfgsmin}
even when memory is not an issue. Remember that both of these algorithms
are \emph{approximating} the Hessian matrix using previous gradient
evaluations. If the true Hessian is changing rapidly, then a limited
memory approximation may be better than a long memory approximation.
The Rosenbrock function is such a case. The program \htmladdnormallink{lbfgsmin-example.m}{http://pareto.uab.es/mcreel/MINTOOLKIT/lbfgsmin-example.m}
minimizes a 200-dimensional Rosenbrock function using both algorithms.
The output\verbatiminput{lbfgsmin-example.out}shows that the limited
memory algorithm uses significantly more iterations that the ordinary
BFGS algorithm, but it is almost 4 times as fast. In general, though,
the ordinary BFGS algorithm is recommended when memory limitations
are not a problem.


\subsubsection{\texttt{samin}}

For discontinuous and/or seriously nonconvex problems, the quasi-Newton
methods are not likely to work well. \texttt{samin} is called as\[
[theta,value,convergence]=samin("f",args,control)\]


The controls for \texttt{samin} are summarized in Table%
\begin{table}

\caption{\label{cap:Controls-for-lbfgsmin}Controls for \texttt{samin}}

\begin{tabular}{|c|c|c|c|}
\hline 
Element&
Name&
Purpose&
Description\tabularnewline
\hline
\hline 
{\footnotesize 1}&
{\footnotesize lb}&
{\footnotesize lower bounds}&
{\footnotesize vector of lower bounds for parameters}\tabularnewline
\hline 
{\footnotesize 2}&
{\footnotesize ub}&
{\footnotesize upper bounds}&
{\footnotesize vector of upper bounds for parameters}\tabularnewline
\hline 
{\footnotesize 3}&
{\footnotesize nt}&
{\footnotesize control looping}&
{\footnotesize loops per temp. reduction, e.g., nt=20}\tabularnewline
\hline 
{\footnotesize 4}&
{\footnotesize ns}&
{\footnotesize control looping}&
{\footnotesize loops per stepsize adjustment, e.g., ns=5}\tabularnewline
\hline 
{\footnotesize 5}&
{\footnotesize rt}&
{\footnotesize reduce temp.}&
{\footnotesize 0 < rt < 1, e.g., rt=0.75}\tabularnewline
\hline
{\footnotesize 6}&
{\footnotesize maxevals}&
{\footnotesize limit evaluations}&
{\footnotesize usually, a large number, unless just exploratory, e.g.,
1e10}\tabularnewline
\hline
{\footnotesize 7}&
{\footnotesize neps}&
{\footnotesize convergence}&
{\footnotesize positive integer. Higher is stricter criterion, e.g.,
neps=5}\tabularnewline
\hline
{\footnotesize 8}&
{\footnotesize functol}&
{\footnotesize convergence}&
{\footnotesize last neps function values must be this close to eachother}\tabularnewline
\hline
{\footnotesize 9}&
{\footnotesize paramtol}&
{\footnotesize convergence}&
{\footnotesize width of search interval must be less than this value}\tabularnewline
\hline
{\footnotesize 10}&
{\footnotesize verbosity}&
{\footnotesize output}&
{\footnotesize 0: no outout; 1: intermediate; 2: only final}\tabularnewline
\hline
{\footnotesize 11}&
{\footnotesize minarg}&
{\footnotesize arg. for min.}&
{\footnotesize which arg to min. w.r.t., usually = 1}\tabularnewline
\hline
\end{tabular}
\end{table}


The example program \htmladdnormallink{sa-example.m}{http://pareto.uab.es/mcreel/MINTOOLKIT/sa-example.m}
is listed here:\verbatiminput{sa-example.m}The objective function
is the sum of $k$ exponentiated cosine waves, each shifted down so
the minimum is zero, with some curvature added in to create a global
minimum of $f(x)=0$ at $x=(0,0,...,0).$ The (edited to shorten)
output of the example is here:\verbatiminput{sa-example.out}

You can see that the minimum was found correctly.


\subsubsection{\label{sub:A-more-difficult}A more difficult problem}

The Mor�-Garbow-Hillstrom test suite contains some relatively difficult
minimization problems. \texttt{bfgsmin} by itself can solve some of
these problems, but not all of them, since some have multiple local
minima, or completely flat regions where a gradient-based method will
not be able to find a decreasing direction of search. The ''\htmladdnormallink{Biggs EXP6}{http://www.uni-graz.at/imawww/kuntsevich/solvopt/results/moreset.html}''
problem \#18 is one for which \texttt{bfgsmin} fails to find the global
minimum. \htmladdnormallink{This program}{http://pareto.uab.es/mcreel/MINTOOLKIT/solvebiggs.m}
shows how the global minimum may be found by combining an initial
search that uses \texttt{samin} to find good starting values with
refinement using \texttt{bfgsmin} to sharpen up the final results.
The \texttt{samin} results from running this program, which use a
fast temperature reduction and a fairly low limit on function evaluations
are:

\begin{singlespace}
\texttt{NO CONVERGENCE: MAXEVALS exceeded}

\texttt{Stage 2, decreasing temperature}

\texttt{Obj. fn. value 0.000006}

\texttt{parameter search width}

\texttt{9.844228 0.000000}

\texttt{4.294696 0.000000}

\texttt{-5.231675 0.000000}

\texttt{-3.114330 0.000000}

\texttt{1.076916 0.000000}

\texttt{1.118343 0.000000}
\end{singlespace}

Then come the BFGS iterations to sharpen up the results. The final
BFGS results are:

\begin{singlespace}
\texttt{BFGSMIN intermediate results: Iteration 33}

\texttt{Stepsize 0.0000000}

\texttt{Using analytic gradient}

\texttt{Objective function value 0.0000000000}

\texttt{Function conv 1 Param conv 1 Gradient conv 1}

\texttt{params gradient change}

\texttt{10.0000 0.0000 0.0000}

\texttt{4.0000 -0.0000 -0.0000}

\texttt{-5.0000 0.0000 0.0000}

\texttt{-3.0000 -0.0000 0.0000}

\texttt{1.0000 -0.0000 0.0000}

\texttt{1.0000 0.0000 0.0000}
\end{singlespace}

\begin{itemize}
\item The minimum is found, but note that the solution values are in a different
order than those given on the SolvOpt web page, with some negative
signs. This problem suffers from a lack of identification - there
are multiple values that give exactly the same value of zero.
\end{itemize}
An alternative which will often be faster, but is less sure to find
the global minimum, is to call \texttt{bfgsmin} with many random starting
values and a limited number of iterations. This is implemented in
\texttt{battery.m}. You can see an example in \htmladdnormallink{This program}{http://pareto.uab.es/mcreel/MINTOOLKIT/solvebiggs2.m}.
This leads to the results

\begin{singlespace}
\texttt{BFGSMIN intermediate results: Iteration 130}

\texttt{Stepsize 0.0000000}

\texttt{Using analytic gradient}

\texttt{Objective function value 0.0000000000}

\texttt{Function conv 1 Param conv 1 Gradient conv 1}

\texttt{params gradient change}

\texttt{4.0000 -0.0000 0.0000}

\texttt{10.0000 0.0000 -0.0000}

\texttt{3.0000 0.0000 0.0000}

\texttt{5.0000 -0.0000 0.0000}

\texttt{1.0000 -0.0000 0.0000}

\texttt{1.0000 0.0000 0.0000}
\end{singlespace}

The minimum is found correctly, and you can see that the problem is
not identified.


\subsubsection{Tips for successful minimization}


\paragraph{Scaling}

Scaling the data and other constant parameters of the objective function
so that the elements of the gradient are of approximately the same
order of magnitude will help improve accuracy of the Hessian approximation.
This can help a lot in obtaining convergence, and the results will
have higher accuracy. \htmladdnormallink{This program}{http://pareto.uab.es/mcreel/MINTOOLKIT/tips.m}
illustrates. The output is \verbatiminput{tip1.out}You can see that
the scaled data gives a more accurate solution, using less than half
the iterations.


\paragraph{Bullet-proofing}

Writing your objective function so that it cannot return NaN or otherwise
crash can save a lot of grief. Insert something like \texttt{if (((abs(obj\_value)
== Inf)) || (isnan(obj\_value))) }

\texttt{obj\_value = realmax;}

\noindent \texttt{endif}

\noindent at the end of the objective function, and then return \texttt{obj\_value}.
This way, parameter values that lead to crashes are penalized, and
will be avoided automatically.


\subsection{Numeric differentiation}

\texttt{numgradient} and \texttt{numhessian} can be used for numeric
differentiation. \texttt{numgradient} returns the derivative of an
$n\times1$ vector-valued function with respect to a $k\times1$ vector
in a $n\times k$ matrix. \texttt{numhessian} returns the derivative
of a real-valued function with respect to a $k\times1$ vector in
a $k\times k$ matrix. Both functions are quite accurate. \htmladdnormallink{numderivatives.m}{http://pareto.uab.es/mcreel/MINTOOLKIT/numderivatives.m},
which follows, shows how it can be done.\verbatiminput{numderivatives.m}

The results are:\verbatiminput{numderivative.out}


\section{Testing the code}

The program \htmladdnormallink{mgh-test.m}{http://pareto.uab.es/mcreel/MINTOOLKIT/mgh-test.m}
allows testing the algorithms using the Mor�-Garbow-Hillstrom test
suite, obtained from the \htmladdnormallink{SolvOpt}{http://www.uni-graz.at/imawww/kuntsevich/solvopt/}
source code. You can compare the output with \htmladdnormallink{these results}{http://www.uni-graz.at/imawww/kuntsevich/solvopt/results/table1.html},
if you like. Note that simply applying BFGS with a single start value
will sometimes lead to a failure of convergence, or convergence to
a non-global minimum. This is expected, considering the nature of
the problems. See section \ref{sub:A-more-difficult} for an appropriate
means of proceeding with these problems.

If you find any bugs in the code, please contact me.

\begin{thebibliography}{5}
\bibitem{key-5}Eaton, J.W., \url{http://www.octave.org/}
\bibitem{key-7}Liu and Nocedal, \url{http://www.ece.northwestern.edu/~nocedal/PDFfiles/limited-memory.pdf}
\bibitem{key-4}Mittelmann, \url{http://plato.asu.edu/topics/problems/global.html}
\bibitem{key-3}Nocedal (1992), \url{http://www.ece.northwestern.edu/~nocedal/PDFfiles/acta.pdf}
\bibitem{key-2}Goffe, \url{http://www.netlib.no/netlib/opt/simann.f}
\bibitem{key-6}Goffe, \char`\"{}SIMANN: A Global Optimization Algorithm using Simulated
Annealing \char`\"{} Studies in Nonlinear Dynamics \& Econometrics,
Oct96, Vol. 1 Issue 3. \end{thebibliography}

\end{document}
